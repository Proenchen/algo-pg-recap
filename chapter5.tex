\section{Separatoren in planaren Graphen}

\textbf{Definition}: Eine Menge $S\subset V$ heißt Separator von $G = (V, E)$, falls der durch $V\setminus S$ induzierte Subgraph von $G$ unzusammenhängend ist.

\begin{center}
	\includegraphics[width=0.2\textwidth]{images/separator.png}
\end{center}
\bigskip
\textbf{Minimum-Balanced-Separator-Problem}: Gegeben sei ein Graph $G = (V, E)$. Finde eine Partition von $V$ in drei Mengen $V_1, V_2$ und $S$, wobei der Separator $S$ minimale Kardinalität hat und $V_1$ von $V_2$ trennt mit $|V_1|,|V_2|\leq\alpha\cdot|V|$ und $\frac{1}{2}\leq\alpha<1$ konstant.
\begin{itemize}
	\item Separator soll also klein sein
	\item Separator soll etwa gleich große Teilgraphen erzeugen
	\item Problem ist NP-schwer
\end{itemize}
\bigskip
\textbf{Planar-Separator-Theorem}: Die Knotenmenge eines zusammenhängenden, planaren Graphen $G = (V, E)$, $n = |V| \geq 5$, kann so in drei Mengen $V_1, V_2, S\subseteq V$ partitioniert werden, dass
\begin{itemize}
	\item $|V_1|,|V_2|\leq\frac{2}{3}\cdot n$
	\item $S$ ist ein Separator, der $V_1$ von $V_2$ trennt
	\item $|S| \leq 4\cdot\sqrt{n}$
\end{itemize}
Eine solche Partition kann in $\mathcal{O}(n)$ Zeit konstruiert werden.

Für den Beweis dieses Satzes benötigen wir folgendes Lemma.\\

\textbf{Lemma}: Sei $G = (V, E)$ ein planarer, zusammenhängender Graph mit $n = |V| \geq 5$ und $T=(V,E(T))$ ein Spannbaum von $G$ mit Wurzel $w$ und Höhe $h$. Die Knotenmenge von $G$ kann so in drei Mengen $V_1, V_2$ und $S$ partitioniert werden, dass
\begin{itemize}
	\item $|V_1|,|V_2|\leq\frac{2}{3}\cdot n$
	\item $S$ ist ein Separator, der $V_1$ von $V_2$ trennt
	\item $|S|\leq 2\cdot h+1$
\end{itemize}
Eine solche Partition kann in $\mathcal{O}(n)$ Zeit konstruiert werden.

\textit{Beweis}: 
\begin{itemize}
	\item Konstruiere eine Triangulierung von $G$. Nach Satz von Euler hat der neue Graph $3n-6$ Kanten und $2n-4$ Facetten.
	\item Spannbaum $T$ von $G$ ist Spannbaum des triangulierten Graphen
	\item In $T$ induziert jede Nichtbaumkante $\{x,y\}$ einen Kreis $K_{x,y}$ mit $\leq 2\cdot h+1$ Knoten (maximal $h$ Knoten in beide Richtungen + Wurzel)
	\begin{center}
		\includegraphics[width=0.15\textwidth]{images/pst-1.png}
	\end{center}
	\item Sei $\text{Inneres}(K_{x,y})$ die Knoten, die innerhalb des Kreises, aber nicht auf dem Rand des Kreises liegen. Definiere $\text{Äußeres}(K_{x,y})$ dementsprechend.
	\item Wähle Nichtbaumkante $\{x,y\}$ aus, wobei $|\text{Inneres}(K_{x,y})|\geq|\text{Äußeres}(K_{x,y})|$
	\item Wennn $|\text{Inneres}(K_{x,y})| \leq \frac{2}{3}n$, dann gilt das Lemma und wir sind fertig
	\item Sei also  $|\text{Inneres}(K_{x,y})|<\frac{2}{3}n$, dann ist $|\text{Äußeres}(K_{x,y})|<\frac{1}{3}n$
	\item \underline{Ziel}: Ersetze $\{x,y\}$ durch eine andere Nichtbaumkante, sodass das Innere kleiner wird und das Äußere nicht über $\frac{2}{3}n$ wächst
	\item Da Graph trianguliert, begrenzt Kante $\{x,y\}$ zwei Dreiecke, von denen eins im $\text{Inneres}(K_{x,y})$ liegt $\implies$ Dreieck $x\;y\;t$
	\begin{center}
		\includegraphics[width=0.15\textwidth]{images/pst-2.png}
	\end{center}
\end{itemize}

\begin{wrapfigure}{r}{0.15\textwidth}
	\centering
	\vspace{0pt}
	\includegraphics[width=0.15\textwidth]{images/pst-3.png}
	\includegraphics[width=0.15\textwidth]{images/pst-4.png}
	\vspace{40pt}
	\vspace{-800pt}
\end{wrapfigure}

\underline{Fall 1}: $\{x,t\}\text{ oder } \{t,y\}$ ist eine Baumkante. Ersetze $\{x,y\}$ durch  $\{x,t\}$.
\begin{itemize}
	\item Falls $t\notin K_{x,y}$:
	\begin{itemize}
		\item $|\text{Äußeres}(K_{x,t})|=|\text{Äußeres}(K_{x,y})|$
		\item $|\text{Inneres}(K_{x,t})|=|\text{Inneres}(K_{x,y})|-1$
	\end{itemize}
	\item Falls $t\in K_{x,y}$:
	\begin{itemize}
		\item $|\text{Äußeres}(K_{x,t})|=|\text{Äußeres}(K_{x,y})|+1$
		\item $|\text{Inneres}(K_{x,t})|=|\text{Inneres}(K_{x,y})|$
	\end{itemize}
\end{itemize}
\bigskip
\underline{Fall 2}: $\{x,t\}\text{ und } \{t,y\}$ sind beides Nichtbaumkanten.
\begin{itemize}
	\item Sei $|\text{Inneres}(K_{x,t})|\geq |\text{Inneres}(K_{t,y})|$. Ersetze $\{x,y\}$ durch  $\{x,t\}$.
	\begin{center}
		\includegraphics[width=0.15\textwidth]{images/pst-5.png}
	\end{center}
	\item $|\text{Äußeres}(K_{x,t})|\leq n-(|\text{Inneres}(K_{x,t})|+P)\leq n-\frac{1}{2}|\text{Inneres}(K_{x,y})|< n-\frac{1}{2}\cdot\frac{2}{3}n=\frac{2}{3}n$
	\item $|\text{Inneres}(K_{x,t})|\leq|\text{Inneres}(K_{x,y})|-1$
\end{itemize}
\bigskip
In beiden Fällen verkleinern wir $|\text{Inneres}(K_{x,y})|$ und lassen $|\text{Äußeres}(K_{x,y})|$ klein genug. Dies kann nun so lange wiederholt werden, bis auch $|\text{Inneres}(K_{x,y})| \leq \frac{2}{3}n$ gilt.

$\implies$ Partition mit den gewünschten Eigenschaften lässt sich konstruieren. Wir müssen nun noch deren Implementation in linearer Laufzeit sicherstellen.

\begin{itemize}
	\item Ersetzung einer Nichtbaumkante durch eine andere, welche die Anzahl der Dreiecke im Inneren reduziert $\implies$ Höchstens $2n-4$ Schritte
	\item In Fall 1 können wir $|\text{Inneres}(K_{x,y})|$ und $|\text{Äußeres}(K_{x,y})|$ in $\mathcal{O}(1)$ berechnen
	\item Für Fall 2 muss entschieden werden, ob $|\text{Inneres}(K_{x,t})|$ oder $|\text{Inneres}(K_{t,y})|$ größer ist. Zeige, dass auch dieser Fall nur konstante Zeit benötigt mithilfe einer amortisierten Analyse.
	\item Führe dazu folgende Vorberechnung durch:
	\begin{itemize}
		\item Durchlaufe $T$ von den Blättern zur Wurzel
		\item Speichere für jeden Knoten $v$ und inzidente Baumkanten die Anzahl Knoten im Unterbaum links bzw. rechts der Kante
		\item Dies kann einmalig in Linearzeit durchgeführt werden
	\end{itemize}
	\item Laufe von $x$ und $y$ abwechselnd in Richtung Wurzel bis erstmals $v$, d.h. Weg von $t$ zur Wurzel, erreicht wird.
	\begin{center}
		\includegraphics[width=0.4\textwidth]{images/pst-6.png}
	\end{center}
	\item $|\text{Inneres}(K_{x,t})|=D+B$
	\item $|\text{Inneres}(K_{t,y})|=A-D-B-W$
\end{itemize}
\bigskip
Die Anzahl der Operationen in einem Schritt ist proportional zu der Anzahl der Knoten in dem Teil von $K_{x,y}$, der nicht weiter betrachtet wird. Also ist auch Fall 2 in amortisiert konstanter Zeit implementierbar.

Damit ist auch die Laufzeit und somit das gesamte Lemma bewiesen.\\

\textbf{BFS-Lemma}: Sei $T = (V,E(T))$ ein BFS-Baum von $G = (V , E)$. Für eine Nichtbaumkante $\{u,v\}$ gilt $|\text{level}(u)-\text{level}(v)|\leq 1$.\\

\textit{Beweis des Planar-Separator-Theorem}: 
\begin{itemize}
	\item Konstruiere eine Triangulierung von $G$ und ein BFS-Baum $T$ mit beliebiger Wurzel
	\item Sei $\mu$ das Level mit der Eigenschaft:
	\begin{center}
		\includegraphics[width=0.26\textwidth]{images/pst-7.png}
	\end{center}
	\item Wenn $|\text{level } \mu|\leq 4\sqrt{n}$, dann ist $\mu$ ein geeigneter Separator und wir sind fertig.
	\item Sei also $|\text{level } \mu|> 4\sqrt{n}$.
	\item Sei $m$ das unterste Level oberhalb von $\mu$ und $M$ das oberste Level unterhalb von $\mu$ mit $|S_m|\leq\sqrt{n}$ und $S_M\leq\sqrt{n}$.
	\begin{center}
		\includegraphics[width=0.3\textwidth]{images/pst-8.png}
	\end{center}
	\item Offensichtlich gilt $|A_1|\leq\frac{n}{2} \text{ und auch }|A_3|\leq\frac{n}{2}$, da schon $>\frac{n}{2}$ Knoten über $\mu$ 
\end{itemize}
\medskip
\underline{Fall 1}: $|A_2|\leq\frac{2}{3}n$
\begin{itemize}
	\item $S=\text{level }m\cup\text{level }M$ ist Separator
	\item $V_1=\max\{A_1,A_2,A_3\}, |V_1|\leq\frac{2}{3}n$
	\item $V_2=V\setminus(S\cup V_1), |V_2|<\frac{2}{3}n$
	\item Damit wurde ein geeigneter Separator gefunden und wir sind fertig
\end{itemize}

\underline{Fall 2}: $|A_2|>\frac{2}{3}n$
\begin{itemize}
	\item Verschmelze die Knoten in $A_1\cup S_m$ zu einem Knoten $s$  und entferne alle Knoten aus $S_M\cup A_3$. Dadurch entsteht ein neuer Graph $G'$.
	\begin{center}
		\includegraphics[width=0.25\textwidth]{images/pst-9.png}
	\end{center}
	\item BFS-Baum $T$ induziert BFS-Baum $T'$ in $G'$
	\item $T'$ hat maximal Höhe $\sqrt{n}$, da $|V'|\leq n$ und durch die Wahl von $m$ und $M$ für jede Schicht $S_i$ zwischen $m$ und $M$ $|S_i|>\sqrt{n}$ gilt
	\item Wende obiges Lemma auf $G'$ und $T'$ an und erhalte $S', U_1, U_2$
	\begin{center}
		\includegraphics[width=0.25\textwidth]{images/pst-10.png}
	\end{center}
	\item Sei $S=S'\cup\text{level } m\cup\text{level } M$
	\item Nach dem Lemma folgt $|S'|\leq 2\sqrt{n}+1$, also $|S|\leq4\sqrt{n}$
	\item Sei $V_1=\max\{U_1,U_2\}.$ Nach dem Lemma gilt $|V_1|\leq\frac{2}{3}n$
	\item Weiterhin gilt $|V_1|+|S|>|V_1|+|S'|>\frac{1}{2}\cdot A_2$. Setzt man also $V_2=V\setminus(S\cup V_1)$, dann gilt $|V_2|=n-|V_1|-|S|<n-\frac{1}{2}\cdot |A_2|<\frac{2}{3}n$
	\begin{center}
		\includegraphics[width=0.25\textwidth]{images/pst-11.png}
	\end{center}
\end{itemize}

Auch hier findet man also einen geeigneten Separator, womit das \textbf{Planar-Separator-Theorem} bewiesen ist.

